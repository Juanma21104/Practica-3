
\documentclass[11pt]{article}
\usepackage[utf8]{inputenc}
\usepackage[english]{babel}
\usepackage{amsthm, amsmath}
\usepackage{nccmath} %Para centrar ecuaciones
\usepackage{graphicx}
\usepackage{enumitem}
\usepackage{algorithmic}
\usepackage{whilecode2}

\graphicspath{ {Images/} }
    \title{\textbf{Práctica 3}}
    \author{Juan Manuel Cardeñosa Borrego}
    \date{}
    
    \addtolength{\topmargin}{-3cm}
    \addtolength{\textheight}{3cm}
\begin{document}

\maketitle
\thispagestyle{empty}

\section*{Ejercicio 1}
Define the TM solution of exercise 3.4 of the problem list and test its correct
behaviour.

\begin{figure}[htp]
\centering
\includegraphics[scale=0.30]{/home/juanna/Escritorio/Juanma/aaUma/1º Cuatri/TALF/Práctica 3/Ejercicio1.png}

\end{figure}

\section*{Ejercicio 2}
Define a recursive function for the sum of three values.\\

\begin{center}
addition\_3 = $<<\pi^1_1|\sigma\left(\pi^3_3\right)>|\sigma\left(\pi^4_4\right)>$
\end{center}

\newpage
\section*{Ejercicio 3}
Implement a WHILE program that computes the sum of three values. You must use an auxiliary variable that accumulates the result of the sum.


\begin{whilecode}[H]

 \While{$X_3 \not = 0$}{

  $X_1 \Assig X_1 + 1$\;
  $X_3 \Assig X_3 - 1$\

 }
 
  \While{$X_2 \not = 0$}{

  $X_1 \Assig X_1 + 1$\;
  $X_2 \Assig X_2 - 1$\

 }
 \DefaultVar{1}\Assig\DefaultVar{1}
\end{whilecode}


\end{document}

